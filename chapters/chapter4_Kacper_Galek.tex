\newpage
\section{Kacper Galek}
\subsection{Wyrażenie matematyczne}


przedstawiamy przykładowe wyrażenie matematyczne:

$$e=\sum_{n=0}^{\infty} \frac{1}{n!}.$$

\subsection{zdjecie}

\includegraphics[width=0.6\textwidth]{pictures/mysliciel.jpg}

\subsection{tabela}

~\ref{tab:table4} prezentuje numery(umiescilem u gory strony)
\begin{table}[]
\begin{tabular}{lllll}
1    & 4    & 45    & 242 & 44   \\
4324 & 2545 & 63444 & 3   & 345  \\
54   & 535  & 5354  & 353 & 54   \\
25   & 566  & 35    & 251 & 6364
\end{tabular}
\label{tab:table4}
\end{table} 

\subsection{Listy}

\begin{enumerate}
    \item to jest
    \item lista 
    \item numerowana
\end{enumerate}

\begin{itemize}
    \item to jest
    \item lista
    \item nienumerowana
\end{itemize}

\subsection{Tekst w dwóch akapitach}
moje ulubione dania to \textbf{pizza i makaron} 
moja ulubiona \underline{pizza} to \textit{pepperoni}. Jest idealnym polaczeniem smakow.\par

moj ulubiony \underline{makaron} to \textit{makaron z owocami morza}. tak jak \textit{pepperoni} jest idealnym polaczeniem smakow

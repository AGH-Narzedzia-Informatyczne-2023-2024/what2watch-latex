\newpage
\section{Jakub Groblicki}

\subsection{Wyrażenie Matematyczne}

Oto przykład wyrażenia matematycznego:

\begin{equation}
    E=mc^2
\end{equation}

\subsection{Zdjęcie}

Poniżej znajduje się przykład wstawienia obrazu:

\begin{figure}
    \centering
    \includegraphics[width=0.6\textwidth]{pictures/skryba.jpg}
    \caption{Jak to jest być skrybą?}
    \label{fig:skryba}
\end{figure}

\subsection{Tabela}

Na górze strony znajduje się przykładowa tabela

\begin{table}
    \centering
    \begin{tabular}{|c|c|c|}
        \hline
        Kolumna 1 & Kolumna 2 & Kolumna 3 \\
        \hline
        Wiersz 1 & A & 123 \\
        Wiersz 2 & B & 456 \\
        Wiersz 3 & C & 789 \\
        \hline
    \end{tabular}
    \caption{Tabela 5}
    \label{tab:table5}
\end{table}


\subsection{Lista Numerowana i Nienumerowana}

Przykłady list:

\begin{enumerate}
    \item Pierwszy element listy numerowanej
    \item Drugi element listy numerowanej
\end{enumerate}

\begin{itemize}
    \item Pierwszy element listy nienumerowanej
    \item Drugi element listy nienumerowanej
\end{itemize}

\subsection{Krótki Tekst z Formatowaniem}

Oto przykład krótkiego tekstu z formatowaniem:

\textbf{Tekst pogrubiony.} \emph{Tekst kursywą.}

\subsubsection{Podrozdział}

To jest treść podrozdziału.

\subsubsection{Inny Podrozdział}

Kolejny tekst w innym podrozdziale.

\subsection{Odwołania do Figur i Tabel}

Możemy teraz odwołać się do wcześniej dodanej figury i tabeli. Figura \ref{fig:skryba} przedstawia zdjęcie, a tabela \ref{tab:table5} to nasza wcześniej utworzona tabela.


